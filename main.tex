\documentclass[a4paper,12pt]{article}

% Page layout
\usepackage[a4paper,margin=2.5cm]{geometry}
\usepackage[utf8]{inputenc}
\usepackage{hyperref}
\usepackage{newtxtext,newtxmath}  % nicer serif font like a journal
\usepackage{microtype}            % better spacing
\usepackage{setspace}             % control line spacing
\usepackage{graphicx}

\usepackage[
    backend=biber,
    style=numeric,
    sorting=none
]{biblatex}


\addbibresource{kilder.bib}


\title{Music and Inner Imagery}
\author{Vanilla Riis Mortensen}
\date{}

\begin{document}

\title{Music and Inner Imagery}
\author{Vanilla Riis Mortensen}
\date{} % empty = no date
\maketitle

% \begin{abstract}
% Summary of the mini-project: aim, method,
% and the key idea of the hypotheses/results.
% \end{abstract}\\

\noindent \textbf{Keywords:} music, visual mental imagery, emotion, mind-wandering

\section{Background}

Visual imagery has been largely neglected within music psychology research despite importance for understanding music-evoked emotions\cite{Review}. In relation to music, visual mental imagery refers to the mechanism through which music evokes internal images in the listener.\cite{Taruffi_2017} 

Large areas still remain almost unexplored, one of them being the correlation of emotion and visual imagery. Research suggests that mental imagery and emotional processes during music listening are closely intertwined\cite{Review}, causing us to ask the following questions:

\begin{itemize}
    \item Does the type of emotion matter?
    \item Does emotion come before imagery or do they occur at the same time?
    \item Are imagery and emotion always connected in the same way?
\end{itemize}

\subsection{Effects of Musical Structure and Emotional Tone on Visual Imagery and Mind-Wandering}

A study by Dahl et al.from 2022\cite{Dahl} looks at what kind of visual mental imagery people experience when listening to music, and how musical structure (tempo, timbre, harmony, instrumentation) influences what people visualize.

135 participants were randomly assigned to listen to one of four 2-minute songs with different characteristics. Participants were then asked if they experienced visual imagery. If yes, they were asked to describe it freely. Then complete a card-sorting task choosing descriptors for the scene (e.g., “in the forest,” “tribal,” “sunny,” etc.) and rate emotions (valence, tension, nostalgia, etc.). 

The songs Deep Blue Day and the Test Song, which had similar characteristics (calm, major key, slow tempo), created peaceful landscapes and positive imagery, and were rated as high valence and low tension. 
The song Tanca, which had characteristics like minor mode, rhythmic, and darker timbres, created imagery of humans, rituals, battles, fire and conflict and were rated as as low valence and high tension. 
The Modified Song with a mixture of TST + TNC characteristics created blended imagery that transitioned from peaceful to ominous and was rated as intermediate. 

The authors concluded that musical and acoustical characteristics strongly influence the imagery people experience, and people listening to the same music tend to visualize similar scenes. \\


\noindent Tarrufi et al. explores the question of whether sad music and happy music make our minds wander differently in their study from 2017\cite{Taruffi_2017}. They did three experiments. 

The first experiment was an online study with 216 participants. Participants listened to sad or happy instrumental music with their eyes closed. After each piece, they answered questions about mind-wandering strength (“Was your attention on the music or somewhere else?”), meta-awareness (“How aware were you of your own thoughts?”), content of thoughts (e.g. about the past/future, self-referential, movement) and form (words vs. imagery). Because sad music is usually slower and happy music faster, they repeated the experiment on 140 participants where they tested sad slow vs happy slow and sad fast vs happy fast. This let them test whether the emotion or the tempo drives mind-wandering. In the third experiment, 24 participants were fMRI-scanned while listening to 4 minutes of sad music and 4 minutes of happy music. The music was matched for tempo and loudness to ensure that the results were not affected by acoustic features but came from the emotional quality of the music (sad vs. happy). 

The study revealed several findings. Sad music leads to inward, personal, reflective thinking, while happy music keeps people externally oriented and focused. The form of thoughts was dominantly imagery for both sad and happy music. Even when the tempo was the same, sadness still produced more mind wandering. Sad music increased activity in the core nodes of the Default Mode Network which is the network underlying daydreaming, self-reflection, autobiographical memory and internally generated thought. \\

%Together, these studies demonstrate that musical features and emotional tone not only shape what we see in our mind but also how our mind works while we are listening.

\subsection{Does Imagery Cause Emotion, or Vice Versa? Evidence From Recent Studies}

In 2008, Juslin and Västfjäll proposed one of the most influential theories about how music triggers emotions.\cite{Review} They explained that music can induce emotion through multiple psychological mechanisms where visual imagery is one of the ways music makes people feel emotions. This means that the music is not directly creating the emotional feeling; instead, the image does. According to them, the music stimulates mental images such as landscapes, people, memories, color,s and narratives. These images themselves carry emotional content (positive, nostalgic, sad, uplifting) which spills over into your own emotional state.\cite{Juslin_Vastfjall_2008}\\

\noindent Day and Thompson challenged Juslin and Västfjäll's theory that imagery causes emotion with their study in 2019\cite{Day_Thompson_2019}. In this study, 49 participants listened to 30 different short pieces of music, each 20 seconds long, taken from classical or pop genres, and were told to press a key as soon as they recognized an emotion in the music, felt an emotional reaction, and experienced a visual mental image. They found that reaction times were fastest for recognizing an emotion, second fastest for feeling the emotion, and slowest for experiencing a visual image. This suggests that people typically feel emotions before images appear.
However it is important to note that they measured simple, "low-level" emotions like pleasantness, energy, and tension. More complex emotions like nostalgia, which mixes happiness, sadness, longing, ect might need more time to develop and could interact differently with imagery.\cite{Review}\\

\noindent In a study by Vroegh from 2018 (presented as a poster)\cite{Vroegh}, they studied the direction of the relationship between imagery and emotion. They did an online survey with 602 participants which listened to music and reported their emotional responses, how vivid/frequent their imagery was, and their attentional focus. Then Vroegh used structural equation modeling to test the causal relationships.
They found that there were two different patterns depending on the emotion type. When emotions were clearly positive, they would lead to imagery. However, when emotions were “mixed” (bittersweet, nostalgic, ambivalent), the imagery actually influenced the emotional response. This suggests that the relationship is not one-directional. It changes depending on whether emotions are simple and clear (e.g., happy) or complex (e.g., nostalgia, bittersweetness).

% The study revealed several findings:
% Finding 1: Sad music causes more mind-wandering as participants’ attention drifted significantly more during sad music than happy music. Participants also had lower meta-awareness during sad music, meaning they were less aware that they were drifting.
% Finding 2: Sad-music thoughts were more self-focused. Thoughts during sad music contained more self-referential ideas, emotional content, nature imagery, mixed affect (sad + positive themes). Thoughts during happy music contained more movement, unknown people (e.g., imagining crowds dancing), attention to the music and experiment and positive emotional tone. 
% Finding 3: The form of thoughts was dominantly imagery for both sad and happy music. Participants overwhelmingly reported visual imagery over inner language (words).
% Finding 4: Even when the tempo was the same, sadness still produced more mind wandering.
% Finding 5: fMRI — Sad music increased activity in the core nodes of the Default Mode Network which is the network underlying daydreaming, self-reflection, autobiographical memory and internally generated thought. 


\subsection{Hypotheses}

The question I want to further explore is whether visual mental imagery comes before emotion, whether emotion comes before imagery, or whether the two processes co-occur during music listening. The findings across studies suggest that the answer may depend on the type of emotional response the music evokes.

Day and Thompson (2019) found that people generally feel an emotion before forming a visual image, but their study focused on simple emotions. In contrast, Vroegh (2018) showed that the direction of influence can reverse for more complex or mixed emotions: when emotions are clearly positive, they tend to lead to imagery, but when emotions are mixed or bittersweet, imagery may instead shape the emotional response. Taruffi et al. (2017) further demonstrated that sad music promotes inward, self-referential mind-wandering, while happy music keeps listeners outwardly focused, even when tempo is controlled.

Taking together these studies, it is possible that happy music with simple, positive emotions produces emotion first, while sad music with more complex emotional states allows imagery to arise first.\\

\noindent Based on this reasoning, the study proposes the following hypotheses:\\

\noindent H1: During sad music, visual mental imagery will occur before emotional experience, reflecting the more complex and self-focused cognitive state evoked by sadness.\\

\noindent H2: During happy music, emotional experience will occur before visual mental imagery, reflecting the simpler and more externally oriented response of happy music.

\section{Methods}

Participants will be recruited through convenience sampling. Each participant listens to two 4-minute excerpts, one sad and one happy, originally used by Taruffi et al. (2017). According to Taruffi et al., these excerpts consist of instrumental film soundtrack music, matched for tempo and loudness, and edited into continuous 4-minute tracks. Using these validated excerpts ensures that differences between conditions stem from emotional tone rather than acoustic features. Participants will listen through high-quality over-ear headphones in a quiet environment. They will be instructed to close their eyes during listening. %as this facilitates emotional engagement and replicates the original study conditions.
The order of conditions will be counterbalanced across participants to control for order effects.

\end{document}
