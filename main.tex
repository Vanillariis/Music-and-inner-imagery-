\documentclass{article}

\usepackage[utf8]{inputenc}
\usepackage{hyperref}

\usepackage[
    backend=biber,
    style=numeric,
    sorting=none
]{biblatex}


\addbibresource{kilder.bib}


\title{Music and Inner Imagery}
\author{Vanilla Riis Mortensen}
\date{}

\begin{document}

\maketitle

\section{Background}

In relation to music, visual mental imagery refers to the mechanism whereby music stimulates internal images in the listener consisting of pictorial representations (e.g., natural land- scape or colors), embodied image schemata (e.g., picturing a melodic movement as an ascending or descending image), or complex visual narratives (e.g., similar to that of a movie).\cite{Taruffi_2017} In the the theoretical framework proposed by Juslin and Västfjäll six mechanisms are identified as through which music listening evokes emotions and moods, one of them being visual imagery.\cite{Juslin_Vastfjall_2008} However, visual imagery have been largely neglected by music psychology research despite their crucial role in music-evoked emotions.\cite{Day_Thompson_2019}\cite{Vuoskoski} Taruffi et al. observed that spontaneous thoughts occur more significantly in the form of images compared with words during both sad and happy music.\cite{Taruffi_2017}

Does emotion function as a mediating mechanism between music and images? 


In relation to music, visual mental imagery refers to the mechanism whereby music
stimulates internal images in the listener consisting of pictorial representations
(e.g., natural landscapes or colors), embodied image schemata (e.g., picturing a
melodic movement as an ascending or descending image), or complex visual
narratives (e.g., similar to that of a movie).

\section{Hypotheses}

H1: Sad music will generate imagery before emotion.

H2: Happy music will generate emotion before imagery.

\end{document}
